

\documentclass{article}


% if you need to pass options to natbib, use, e.g.:
%     \PassOptionsToPackage{numbers, compress}{natbib}
% before loading neurips_2023


% ready for submission
\usepackage[preprint]{neurips_2023}


% to compile a preprint version, e.g., for submission to arXiv, add add the
% [preprint] option:
%     \usepackage[preprint]{neurips_2023}


% to compile a camera-ready version, add the [final] option, e.g.:
%     \usepackage[final]{neurips_2023}


% to avoid loading the natbib package, add option nonatbib:
%    \usepackage[nonatbib]{neurips_2023}


\usepackage[utf8]{inputenc} % allow utf-8 input
\usepackage[T1]{fontenc}    % use 8-bit T1 fonts
\usepackage{hyperref}       % hyperlinks
\usepackage{url}            % simple URL typesetting
\usepackage{booktabs}       % professional-quality tables
\usepackage{amsfonts}       % blackboard math symbols
\usepackage{nicefrac}       % compact symbols for 1/2, etc.
\usepackage{microtype}      % microtypography
\usepackage{xcolor}         % colors
\usepackage{amsmath,amssymb}
\usepackage{listings,graphicx}
\usepackage{algorithm}
\usepackage[noend]{algpseudocode}

\title{Report for Programming Assignment 3 (Sudoku)}


% The \author macro works with any number of authors. There are two commands
% used to separate the names and addresses of multiple authors: \And and \AND.
%
% Using \And between authors leaves it to LaTeX to determine where to break the
% lines. Using \AND forces a line break at that point. So, if LaTeX puts 3 of 4
% authors names on the first line, and the last on the second line, try using
% \AND instead of \And before the third author name.


\author{%
  Matthew Callahan\\
  \And
  Suphalerk Lortaraprasert
  % examples of more authors
  % \And
  % Coauthor \\
  % Affiliation \\
  % Address \\
  % \texttt{email} \\
  % \AND
  % Coauthor \\
  % Affiliation \\
  % Address \\
  % \texttt{email} \\
  % \And
  % Coauthor \\
  % Affiliation \\
  % Address \\
  % \texttt{email} \\
  % \And
  % Coauthor \\
  % Affiliation \\
  % Address \\
  % \texttt{email} \\
}


\begin{document}


\maketitle


\begin{abstract}
  
 \end{abstract}

\section{Introduction}

 
\section{Environment Descriptions}
 
 
 
\subsection{Backtracking Search}
 
   
   
   \label{alg:Backtracking Search}
 

\subsection{Constraint Propagation}
  
  \label{alg:Constraint Propagation}
 

\section{Experimental setup}
\subsection{Fixed  Baseline}
\subsection{Most Constrained Variable}

\section{Results}
 \subsection{Backtracking}
\subsection{Runtime Performance}
\begin{table}[h]\centering
  \begin{tabular}{lll}
    \toprule
    puzzle number& Arbitrary Order Backtracks & Most Constrained Variable Backtracks\\
    \midrule
    1& 5 & 0\\
    \midrule
    6& 13&0\\
    \midrule
    7& 10&0\\
    \midrule
    20&10&0\\
    \midrule
    30& 8&0\\
    \midrule
    41& 12& 0\\
    \midrule
    42& 6& 0\\
    \midrule
    43& 13&0\\
    \midrule
    44&9&0\\
    \midrule
    50&6&0\\
    \midrule
    55 &11 &0\\
    \midrule
    61& 11 &0\\
    \midrule
    62 &5&0\\
    \midrule
    67&7&0\\
    \midrule
    68& 5&0\\
    \midrule
    69&7&0\\
    \midrule
    70& 12 &1\\
    \bottomrule
  \end{tabular}
  \caption{Number of backtracks for various algorithms for the easy puzzles}
  \label{tab:resultsEasy}
\end{table}
\section{Discussion}
% Answer the remaining questions not answered in earlier sections



  
\end{document}