\documentclass{article}


% if you need to pass options to natbib, use, e.g.:
%     \PassOptionsToPackage{numbers, compress}{natbib}
% before loading neurips_2023


% ready for submission
\usepackage[final]{neurips_2023}


% to compile a preprint version, e.g., for submission to arXiv, add add the
% [preprint] option:
%     \usepackage[preprint]{neurips_2023}


% to compile a camera-ready version, add the [final] option, e.g.:
%     \usepackage[final]{neurips_2023}


% to avoid loading the natbib package, add option nonatbib:
%    \usepackage[nonatbib]{neurips_2023}


\usepackage[utf8]{inputenc} % allow utf-8 input
\usepackage[T1]{fontenc}    % use 8-bit T1 fonts
\usepackage{hyperref}       % hyperlinks
\usepackage{url}            % simple URL typesetting
\usepackage{booktabs}       % professional-quality tables
\usepackage{amsfonts}       % blackboard math symbols
\usepackage{nicefrac}       % compact symbols for 1/2, etc.
\usepackage{microtype}      % microtypography
\usepackage{xcolor}         % colors
\usepackage{amsmath,amssymb}
\usepackage{listings}
\usepackage{algorithm}
\usepackage[noend]{algpseudocode}

\title{Report for Programming Assignment 1}


% The \author macro works with any number of authors. There are two commands
% used to separate the names and addresses of multiple authors: \And and \AND.
%
% Using \And between authors leaves it to LaTeX to determine where to break the
% lines. Using \AND forces a line break at that point. So, if LaTeX puts 3 of 4
% authors names on the first line, and the last on the second line, try using
% \AND instead of \And before the third author name.


\author{%
  Matthew Callahan\\
  \And
  Suphalerk Lortaraprasert
  % examples of more authors
  % \And
  % Coauthor \\
  % Affiliation \\
  % Address \\
  % \texttt{email} \\
  % \AND
  % Coauthor \\
  % Affiliation \\
  % Address \\
  % \texttt{email} \\
  % \And
  % Coauthor \\
  % Affiliation \\
  % Address \\
  % \texttt{email} \\
  % \And
  % Coauthor \\
  % Affiliation \\
  % Address \\
  % \texttt{email} \\
}


\begin{document}


\maketitle


\begin{abstract}
 This will be finished after we have the results.
\end{abstract}

\section{Introduction}
In this paper we present performance of three different autonomous agents that control a simulated robot vacuum in two different configurations of room. The robot vacuum is assumed to have a perfect sensors and can perfectly navigate and clean individual squares of a 10 by 10 grid. The room begins completly covered in dirt and the goal of the agents is to clean as much of the room as fast as possible. 
\section{Agent Descriptions}
The available actions of the agent are to clean the square it is on, to move forward one square, to turn left (counter clockwise) $90^\circ$, to turn right (clockwise) $90^\circ$, and to turn off. Since there is no cost for making actions, no agent that we designed was instructed to turn off.

The agents have te ability to detect whether the square they are on are clean or dirty with 100\% reliability. Additionally, they can detect if they are facing a wall and if they are on the square they started on. This last sensor input was not determined to be useful for our purposes. 
\subsection{Memoryless deterministic reflex agent}
This agent was designed to have only deterministic rules for actions based solely on the current precepts. When designing this, we determined that the highest priority action was to clean the square it was on if it found the square to be dirty. Otherwise, it would turn if it found itself facing a wall, and advance otherwise. The design is shown explicitly in Algorithm~\ref{alg:SimpleAgent}
\begin{algorithm}[h]
  
  \caption{Programatic Description of Simple Reflex Agent}
  \begin{algorithmic}[1]
    
    \For {timestep $t$ until termination}
    \If {on dirty tile}
    \State clean tile 
    \EndIf
    \If{ not on dirty tile and facing wall}
    \State turn clockwise
    \EndIf
    \If{not facing wall and not on dirty tile}
    \State move forward
    \EndIf
    \EndFor
  \end{algorithmic}
  \label{alg:SimpleAgent}
\end{algorithm}
\subsection{Memoryless random reflex agent}


\subsection{Low-memory deterministic reflex agent}

\section{Experimental setup}

\section{Results}


\section{Discusion}
%Answer the remaining questions not answered in earlier sections
\subsection{Tables}


All tables must be centered, neat, clean and legible.  The table number and
title always appear before the table.  See Table~\ref{sample-table}.


Place one line space before the table title, one line space after the
table title, and one line space after the table. The table title must
be lower case (except for first word and proper nouns); tables are
numbered consecutively.


Note that publication-quality tables \emph{do not contain vertical rules.} We
strongly suggest the use of the \verb+booktabs+ package, which allows for
typesetting high-quality, professional tables:
\begin{center}
  \url{https://www.ctan.org/pkg/booktabs}
\end{center}
This package was used to typeset Table~\ref{sample-table}.


\begin{table}
  \caption{Sample table title}
  \label{sample-table}
  \centering
  \begin{tabular}{lll}
    \toprule
    \multicolumn{2}{c}{Part}                   \\
    \cmidrule(r){1-2}
    Name     & Description     & Size ($\mu$m) \\
    \midrule
    Dendrite & Input terminal  & $\sim$100     \\
    Axon     & Output terminal & $\sim$10      \\
    Soma     & Cell body       & up to $10^6$  \\
    \bottomrule
  \end{tabular}
\end{table}


\end{document}