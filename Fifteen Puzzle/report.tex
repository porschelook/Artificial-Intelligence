\documentclass{article}


% if you need to pass options to natbib, use, e.g.:
%     \PassOptionsToPackage{numbers, compress}{natbib}
% before loading neurips_2023


% ready for submission
\usepackage[preprint]{neurips_2023}


% to compile a preprint version, e.g., for submission to arXiv, add add the
% [preprint] option:
%     \usepackage[preprint]{neurips_2023}


% to compile a camera-ready version, add the [final] option, e.g.:
%     \usepackage[final]{neurips_2023}


% to avoid loading the natbib package, add option nonatbib:
%    \usepackage[nonatbib]{neurips_2023}


\usepackage[utf8]{inputenc} % allow utf-8 input
\usepackage[T1]{fontenc}    % use 8-bit T1 fonts
\usepackage{hyperref}       % hyperlinks
\usepackage{url}            % simple URL typesetting
\usepackage{booktabs}       % professional-quality tables
\usepackage{amsfonts}       % blackboard math symbols
\usepackage{nicefrac}       % compact symbols for 1/2, etc.
\usepackage{microtype}      % microtypography
\usepackage{xcolor}         % colors
\usepackage{amsmath,amssymb}
\usepackage{listings,graphicx}
\usepackage{algorithm}
\usepackage[noend]{algpseudocode}

\title{Report for Programming Assignment 2 (Fifteen Puzzle)}


% The \author macro works with any number of authors. There are two commands
% used to separate the names and addresses of multiple authors: \And and \AND.
%
% Using \And between authors leaves it to LaTeX to determine where to break the
% lines. Using \AND forces a line break at that point. So, if LaTeX puts 3 of 4
% authors names on the first line, and the last on the second line, try using
% \AND instead of \And before the third author name.


\author{%
  Matthew Callahan\\
  \And
  Suphalerk Lortaraprasert
  % examples of more authors
  % \And
  % Coauthor \\
  % Affiliation \\
  % Address \\
  % \texttt{email} \\
  % \AND
  % Coauthor \\
  % Affiliation \\
  % Address \\
  % \texttt{email} \\
  % \And
  % Coauthor \\
  % Affiliation \\
  % Address \\
  % \texttt{email} \\
  % \And
  % Coauthor \\
  % Affiliation \\
  % Address \\
  % \texttt{email} \\
}


\begin{document}


\maketitle


\begin{abstract}
 This will be finished after we have the results.
\end{abstract}

\section{Introduction}

In this paper, we examine the performance of two search algorithms, Compassion A* and RBFS, using a 4x4 puzzle as a representation. The environment is randomized to ensure optimal exploration. To enhance efficiency, we introduce a random algorithm designed to avoid repeating the same steps.

\section{Environment Descriptions}

\subsection{Heuristic function}

\subsection{A* search}

\begin{algorithm}[h]
  
  \caption{Programmatic Description of Simple Reflex Agent}
  \begin{algorithmic}[1]
    
    \For {timestep $t$ until termination}
    \If {on dirty tile}
    \State clean tile 
    \EndIf
    \If{ not on dirty tile and facing wall}
    \State turn clockwise
    \EndIf
    \If{not facing wall and not on dirty tile}
    \State move forward
    \EndIf
    \EndFor
  \end{algorithmic}
  \label{alg:A*}
\end{algorithm}


\subsection{RBFS search}
  \label{alg:RBFS}
\end{algorithm}

\section{Experimental setup}

\section{Results}



\section{Discussion}
% Answer the remaining questions not answered in earlier sections
1. Is there a clear preference ordering among the heuristics you tested considering the number of nodes searched and the total CPU time taken to solve the problems for the two algorithms?

2. Can a small sacrifice in optimality give a large reduction in the number of nodes expanded? What about CPU time?

3. Is the time spent on evaluating the heuristic a significant fraction of the total problem-solving time for any heuristic and algorithm you tested?

4. How did you come up with your heuristic evaluation function?

5. How do the two algorithms compare in the amount of search involved and the cpu-time?

\end{document}